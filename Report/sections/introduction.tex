\documentclass[../main.tex]{subfiles}
 
\begin{document}


Here I will give a brief overview of the mathematics necessary to follow
the workings / implementation of the Simulator, and also to refresh
slightly on the notation and concepts of what will be used throughout
this paper.\\
A working knowledge of linear algebra, matrix operations (\emph{i.e.
multiplication, transpose, trace and tensor products}) among others,
along with a basic idea of some Quantum Mechanics / Quantum
Informational Theory and the workings of Classical Computers, is helpful
but not 100\% necessary. I will try and keep away from the extreme
specifics, and the reasons behind everything, to the best of my
abilities, although some detail is needed in order to portray the ideas
behind this. If you are interested in learning more, the book
``\emph{Quantum Computing for Computer Scientists}'' \cite{quantumcomputerscience} is very good.

\subsection{The Qubit}

In classical computing, the bit (short for binary unit) is the smallest
data unit. It has a \emph{Single binary value, that can be either
\texttt{0} or \texttt{1}}.\\
The \emph{qubit} would be the Quantum version of the classical bit.
However, unlike a classical bit, it can take a superposition, meaning
that it is possible for a qubit to be in a linear combination of
states.\\
In this paper, the choice of basis states (\(Z\)) are as follows:
\begin{equation}
	\lvert0\rangle =\left( \begin{matrix} 1\\ 0 \end{matrix} \right), \lvert1\rangle =\left( \begin{matrix} 0\\ 1\end{matrix} \right)
\end{equation}

The notation used here is called Bra-Ket notation, or Dirac Notation. A
common use for it is to describe quantum state, which is what will be
used it for in this paper.\\
You can think of whats in between the \(\lvert\rangle\) as a label for our
state.\\
The difference between the notation above, and notation like
\(\langle\lvert\), among others, is that one is column vector and the other a
row vector.\\
As I said above, it is also possible for a qubit to be in a
superposition. A single qubit superposition would look like this:\\
\[
\left\lvert \Psi \rangle =\alpha\lvert0\rangle +\beta\right\lvert 1\rangle\]\\
The coefficients - \(\alpha\) and \(\beta\) are complex numbers.

Because this state is a superposition, fundamentally you cannot get
access to the amplitudes of the base vectors \(\alpha\) and \(\beta\).
Instead, you will either get \(\lvert0\rangle\) or \(\lvert1\rangle\), with the
probability of getting \(\lvert0\rangle\) being \(\lvert\alpha\lvert^2\), and the
probability of getting \(\lvert1\rangle\) being \(\lvert\beta\lvert^2\).\\
Also, if you add the probabilities, you will always get \(1\), which is
the normalization.\\
\begin{equation}
	\lvert\alpha\lvert^2 + \lvert\beta\lvert^2 = 1
\end{equation}

If an operation is being performed on a qubit with multiple states, the
operation is performed on all of the states at the same time. However,
when observed, the qubit will collapse into either \(0\) or \(1\),
stochastically, based on the probability of each of the basic states.\\
This property can be realised by using partical spin, among others, but
I will not get into that as this project is on a practical
implementation of a quantum computer.\\
Richard Feynman (a theoretical physicist known for his contribution to
quantum mechanics) suggested that a qubit,
\(\alpha\lvert0\rangle + \beta\lvert1\rangle\), occupies all possible states
between \(0\) and \(1\), thereby making it possible to encode an
infinite amount of data. Although, when observed, collapses into a state
of 0 or 1, thus making the data held in the superpositions useless. This
is why \emph{Quantum Algorithms} are used to exploit this trait.

Just like a classical computer, while it is possible to have a single
qubit quantum computer, it's not going to be much use. To use the power
of quantum computing you need to have more then 1 qubit.\\
To define more then 1 qubit, you use a \emph{register}. For example, if
you wanted to put the value \texttt{11010} Onto 5 qubits, you set the
state as shown:\\
\[
\lvert11010\rangle\]\\
which is equal to:\\
\[
\lvert11010\rangle = \lvert1\rangle \otimes \lvert1\rangle \otimes \lvert0\rangle \otimes \lvert1\rangle \otimes \lvert0\rangle\]\\
where \(\otimes\) is the tensor product.\\
A Quantum Register is just a number of qubits, tensored together.\\
An \emph{n-bit} Quantum Register can hold any number of \emph{n-qubit}
states / superpositions.

\subsection{Quantum Logic Gates}

Now that I have defined the qubit, we can now move onto quantum logic
gates.\\
Just like classical logic gates, they work on the register to change the
values (or states) or the bits (or qubits).\\
However, unlike classical computing, all quantum logic gates are
reversible (although you can use reversible gates in classical
computation).\\
Quantum logic gates are represented by \emph{unitary matrices}, and most
logic gates act on either One or Two Qubits. Hence, they can be
described by a \texttt{2x2} or \texttt{4x4} Unitary Matrix.\\
A unitary matrix is defined so that:

\begin{quote}
A Complex, Square Matrix, \textbf{U} is unitary if its complex conjugate
(U*) is also its inverse
\end{quote}

Also, just to recap, when calculating the result of the gates, you must
remember that states are just vectors, so\\
\[\lvert1\rangle =\left( \begin{matrix} 0\\ 1\end{matrix} \right)\]\\
\[
V_1\lvert0\rangle + V_2\lvert1\rangle =\left( \begin{matrix} V_1\\ V_2\end{matrix} \right)\]\\
and\\
\[
V_1\lvert00\rangle + V_2\lvert01\rangle + V_3\lvert10\rangle + V_4\lvert11\rangle =\left( \begin{matrix} V_1\\V_2\\V_3\\ V_4\end{matrix} \right)\]\\
Now with that defined, I will now show you some logic gates.

\subsubsection{The NOT Gate / Pauli-X Gate}

This is probably the Simplest gate, and it acts very similarly to the NOT gate in classical computing.
This gate acts on a single qubit, and as you can probably tell, flips the value of the qubit.
Because of this it's sometimes called the bit-flip gate.\\
This gate represents a rotation around a
\href{https://en.wikipedia.org/wiki/Bloch_sphere}{Bloch Sphere} by the
\(x\)-axis by \(\pi\) radians.\\
It maps \(\lvert0\rangle\) to \(\lvert1\rangle\), and \(\lvert1\rangle\) to
\(\lvert0\rangle\).\\
And for a superposition, it just flips the bits, i.e.:\\
\[
\alpha\lvert0\rangle + \beta\lvert1\rangle \Rightarrow \alpha\lvert1\rangle + \beta\lvert0\rangle\]\\
This gate can be represented using this matrix:\\
\begin{equation}
	X = \left[ \begin{matrix} 0 & 1\\1 & 0\end{matrix} \right]
\end{equation}

\subsubsection{Pauli-Y Gate}

This gate acts on a single qubit, and works out to a rotation around a
Bloch Sphere on the \(y\) axis by \(\pi\) radians.\\
It maps the basic states as follows:\\
\[
\lvert0\rangle \Rightarrow i\lvert1\rangle, \lvert1\rangle \Rightarrow -i\lvert0\rangle\]\\
Its matrix form:\\
\begin{equation}
	Y = \left[ \begin{matrix} 0 & -i\\i & 0\end{matrix} \right]
\end{equation}

\subsubsection{Pauli-Z Gate}

This gate again acts on a single qubit, and equates to a rotation on a
Bloch Sphere on the \(z\) axis by \(\pi\) radians.\\
It leaves the basic state \(\lvert0\rangle\) Unchanged, and maps the state
\(\lvert1\rangle \Rightarrow -\lvert1\rangle\). It's also sometimes called the
\emph{`Phase Change Gate'} Due to this.\\
Its matrix form:\\
\begin{equation}
	Z = \left[ \begin{matrix} 1 & 0\\0 & -1\end{matrix} \right]
\end{equation}

\subsubsection{The Hadamard Gate}

The purpose of this gate is to map the state of a qubits basic state to
2 superpositions, each of equal weight, which meaning the probability of
each state being chosen is equal.\\
It represents a combination of two rotations around a Bloch Sphere,
\(\pi /2\) about the Y-axis \(\pi\) about the X-axis.\\
It maps the states as shown:\\
\[
\lvert0\rangle \Rightarrow \frac{\lvert0\rangle + \lvert1\rangle}{ \sqrt{2}}                 \lvert1\rangle \Rightarrow \frac{\lvert0\rangle - \lvert1\rangle}{ \sqrt{2}}\]\\
It can represented using the Hadamard matrix:\\

\begin{equation}
	H = \frac{1}{\sqrt{2}} \left[ \begin{matrix} 1 & 1\\1 & -1\end{matrix} \right]
\end{equation}

You can define these states:\\

\begin{equation}
	\lvert+\rangle = \frac{\lvert0\rangle +\lvert1\rangle}{\sqrt{2}}, 
	\lvert-\rangle = \frac{\lvert0\rangle -\lvert1\rangle}{\sqrt{2}}
\end{equation}

Where \(\lvert+\rangle\) and \(\lvert-\rangle\) make up a new basis state called
the \emph{conjugate / diagonal} state, or \(X\)

\subsubsection{The Identity Gate}

This gate simply maps the qubits state from input to output, i.e. It is
unchanged. $|0\rangle \Rightarrow |0\rangle, |1\rangle \Rightarrow |1\rangle$. Its Matrix is:\\
\begin{equation}
	Id = \left[ \begin{matrix} 1 & 0\\0 & 1\end{matrix} \right]
\end{equation}

\subsubsection{The S / Phase Gate}

Another Basis State is the \emph{Circular} Basis, $Y$,
$\lvert\circlearrowright\rangle$ and $\lvert\circlearrowleft\rangle$, with the
states both equating to:\\
\begin{equation}
	\lvert\circlearrowright\rangle = \frac{\lvert0\rangle +i\lvert1\rangle}{\sqrt{2}}, \lvert\circlearrowleft\rangle = \frac{\lvert0\rangle -i\lvert1\rangle}{\sqrt{2}}
\end{equation}
To make these states, you need an a new gate \(S\) or \emph{Phase
Gate}.\\
The Phase Gate's Matrix looks like this:\\
\begin{equation}
	S = \left[ \begin{matrix} 1 & 0\\0 & i\end{matrix} \right]
\end{equation}
When used on a qubit $\lvert1\rangle$ after a Hadamard gate (\(H\)), it
applies a \emph{Complex Phase} (Hence the name, Phase Gate), and you can
get the state \(\lvert\circlearrowright\rangle\).\\
To get the state \(\lvert\circlearrowleft\rangle\), you use the same pattern,
but replacing the gate \(S\) with \(S^\dagger\).\\
You can represent \(S^\dagger\) With the matrix:\\
\begin{equation}
	S^\dagger = \left[ \begin{matrix} 1 & 0\\0 & -i\end{matrix} \right]
\end{equation}

\subsubsection{Multi Qubit Gates}

Up until now, we have only looked at gates that act on a single qubit.\\
It is at this point that you have to remember the exponential growth of
states in qubits, one of the factors that makes it hard to simulate a
quantum computer.\\
In short, the complex vector space for a 5 qubit system has a dimension
equal to \(2^5\) or \(32\). Now if you look at, for example, a 64 qubit
system, you get \(2^64 = 18446744073709551616\), or just under eighteen
and a half quintillion. As you can probably see, in an \(n\)th qubit
system, you need \(2^n\) numbers to represent its state. This growth is
exponential, and if you had a \(300\) qubit system, you would need more
numbers then there are atoms in the universe to represent its state.\\
If you want to see this in effect, look at the standard basis, or all
the possible binary strings. For 2 qubits,\\
\[
\{\lvert00\rangle, \lvert01\rangle, \lvert10\rangle, \lvert11\rangle\}\]\\
or for 3 qubits:\\
\[
\{\lvert000\rangle, \lvert001\rangle, \lvert010\rangle, \lvert011\rangle, \lvert100\rangle, \lvert101\rangle, \lvert110\rangle, \lvert111\rangle,\}\]\\
A Classical computer also needs \(2^n\) Numbers to define what position
the bits are in, but this is much simpler then in Quantum Computing,
where you don't just have a single set state, but rather, due to
Superpositions and Entanglement, the set at any point is very different.
To describe the quantum state, you need \(2^n\) Complex Numbers. Now,
onto some gates.

\subsubsection{CNOT / Controlled Not Gate}

This is gate is an example of Condition Based Logic, which is needed to
do anything interesting in Quantum Computing. This Gate, the CNOT, is
the only one that will be implemented in the simulator, to show the
purpose of Conditional Logic. \\
What this gate does, is if the control qubit is \(\lvert1\rangle\), then it
applies a NOT (\(X\)) Gate to the target qubit, otherwise, it does
nothing. Unlike its classical counterpart, the XOR gate, it is
reversible (Like all quantum gates), and it does this by having 2
outputs.\\
The CNOT's matrix form is:\\
\begin{equation}
	CNOT = \left[ \begin{matrix} 1 & 0 & 0 & 0 \\ 0 & 1 & 0 & 0 \\ 0     & 0 & 0 & 1 \\ 0 & 0 & 1  & 0\end{matrix} \right]
\end{equation}

\subsubsection{T or Pi/8 Gate}

The reason that I have this gate included is because of a few things. So
far, all of the gates we have gone through
\(\{ X,Y,Z,H,S,S^\dagger, CNOT \}\), are all part of whats called a
Clifford Group. Every gate in the Clifford group can be simulated pretty
well on a classical computer (See the Gottesman-Knill Theorem \cite{gottesmanknill}),
and therefore the Clifford group is not \emph{universal} (Basically that
any possible operation on a quantum computer can be reduced to and
preformed on, see
\href{https://en.wikipedia.org/wiki/Quantum_gate\#Universal_quantum_gates}{Here}).
To use the `Quantum Computer' (The simulator) properly, we will need to
include another gate, thats not part of the Clifford Group. For this, we
will use \(T\) and \(T^\dagger\).\\
They can be represented with the matrices:\\
\begin{equation}
	T = \left[ \begin{matrix} 1 & 0\\0 & e^\frac{i\pi}{4}\end{matrix} \right], T^\dagger = \left[ \begin{matrix} 1 & 0\\0 & e^-\frac{i\pi}{4}\end{matrix} \right]
\end{equation}

The \(T\) Gate makes it possible to access all points on a Bloch
Sphere.\\ \\ \\
These all of the gates that will be implemented into the simulator.

\end{document}